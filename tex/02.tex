\section{Алгоритмы с ветвлением}

\subsection{Условный оператор}

\task Три точки на плоскости заданы своими координатами. Написать
программу, определяющую, лежат ли они на одной прямой.

\task Дан шестизначный номер билета. Написать программу, проверяющую,
является ли билет «счастливым». (Билет называется «счастливым», если
сумма первых трёх цифр равна сумме последних трёх цифр.)

\task Стоимость минуты разговора по телефону — $p~\textup{руб.}$ Если
продолжительность разговора превышает 5 минут, то на оставшуюся часть
времени действует скидка, равная 20\,\%. Написать программу,
определяющую стоимость разговора заданной продолжительности $t$.

\task Дано трёхзначное число $a$ и цифра $k$. Написать программу,
проверяющую, содержится ли $k$ в $a$.

\task Написать программу, пределяющую, является ли заданное
четырёхзначное число палиндромом. (Число-палиндром — это число, запись
которого слева направо совпадает с записью справа налево.)

\task Даны радиус окружности и катеты прямоугольного
треугольника. Написать программу определяющую, можно ли вписать
треугольники в окружность.

\task Написать программу, вычисляющую для заданного $n$ значение функции
\[
f(n) =
\begin{cases}
  n/2,  &\textup{если }n\textup{ чётное,} \\
  3n+1, &\textup{если }n\textup{ нечётное.}
\end{cases}
\]

\task Дан прямоугольник размерами $w\times h.$ Написать программу,
определяющую, можно ли полностью покрыть его $n$ плитками размера
$a\times a.$

\task Снаряд выпущен под углом $\alpha$ горизонту с начальной
скоростью $v$. Написать программу, проверяющую, попадёт ли он в цель
высотой $h$, находящуюся на расстоянии $L$ от пушки. (Ускорение
свободного падения $g\approx 9{,}81~\frac{\textup{м}}{\textup{с}^2}$,
сопротивлением воздуха пренебречь.)

\task Дано целое число $k$ ($1 \leqslant k \leqslant 365$). Написать
программу, определяющую, придётся ли $k$-й день года на воскресенье,
если 1 января — понедельник.

\subsection{Составные условия} 

% високосный год


\task Даны 2 числа, определить является ли наименьшее из них чётным.

\task Написать программу определяющую будут ли цифры данного
четырёхзначного числа, записанные в том же порядке, что и в числе,
образовавать возрастающую последовательность.

\task Составить программу, определяющую является ли треугольник
равносторонним.

\task Даны числа a и b. Написать программу, которая определяет делятся
ли эти числа на 3, 5, 7 одновременно.

\task Написать программу, определяющую являются ли a, b и c сторонами
треугольника.

\task Дано пятизначное число, определить содержет ли оно цифры 3 и 4
или 1 и 7.

\task Написать программу, определяющую является ли g остатком от
деления n на k либо k на n.

\task Написать программу, определяющую, лежит ли точка с абсциссой x в
одном из интервалов [-10,5], [10,15].

\task Составить программу, которая определяет делится ли сумма цифр
данного четырёхзначного число на n и при этом является ли она
двузначной.

\subsection{Применение нескольких условных операторов}
% три числа в порядке возрастания

% квадратное уравнение

% медиана трёх чисел

% вид треугльника

% сколько различных трёхзначных чисел можно составить из цифр трёхзначного числа

\subsection{Оператор выбора}

% возрастная категория

% день недели через К дней
