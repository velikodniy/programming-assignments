\section{Файлы и сериализация}

\subsection{Файлы}

\task В текстовом файле хранится таблица действительных чисел. В
каждой строке записаны три числа, разделённые символом «;». Написать
программу, находящую для каждого столбца сумму и арифметическое
среднее.

В случае, если формат данных в файле не соответствует указанному,
вывести сообщение об ошибке.

\task Написать программу, выводящую таблицу частот русских букв (без
учёта регистра) в заданном текстовом файле. Таблицу упорядочить по
убыванию частоты.

\task В текстовом файле записана информация о продолжительности
телефонных звонков.  Информация о каждом звонке записана в отдельной
строке в формате «Фамилия:Продолжительность». Продолжительность
указана в секундах.

Написать программу, находящую суммарную продолжительность разговоров
для каждого из абонентов.

\task Написать функции для сохранения в файл и чтения из файла
двумерного массива целых чисел. В начало файла должны записываться
размеры массива, а затем в двоичном виде элементы построчно.

\task Написать программу, шифрующую файл и записывающую результат в
новый файл. Шифрование выполнить прибавлением по модулю 2 (операция
XOR) к каждому байту исходного файла заданного однобайтового числа.

Замечание: этот метод крайне ненадёжен и не должен использоваться
для обеспечения безопасности данных.

\task

\task 

\task

\task 

\task


\subsection{Файловая система}

\task

\task

\task

\task

\task

\task

\task

\task

\task

\task


\subsection{Сериализация}

\task

\task

\task

\task

\task

\task

\task

\task

\task

\task

